% пакеты
\usepackage{csquotes}
\usepackage[utf8]{inputenc}
%\usepackage[T1]{fontenc}
\usepackage[T2A]{fontenc}
\usepackage{graphicx}
\usepackage[newfloat]{minted}
%\usepackage[many]{tcolorbox}
\usepackage{caption}

\newcommand{\txtmint}[1]{\mintinline[fontsize=\scriptsize]{text}{#1}}

\newenvironment{code}{\captionsetup{type=listing}}{}
\SetupFloatingEnvironment{listing}{name=Листинг}

% Библиография
\NewBibliographyString{accessmode}
\DeclareFieldFormat{url}{\bibstring{accessmode}\addcolon\space\url{#1}}
\DeclareFieldFormat{title}{#1}
\DefineBibliographyStrings{russian}{
accessmode={Режим доступа:},
urlseen={дата обращения:}
}
% скобки после пунктов
\setenumerate[1]{label={ \arabic*)}}
\usepackage{enumitem}

% \FloatBarrier
\usepackage{placeins}

% $ в листингах
\newcommand{\dlr}{\mbox{\textdollar}}

% Тире в itemize
\renewcommand\labelitemi{---}

% \foreach
\usepackage{pgffor}

% Длинные таблицы
\usepackage{longtable}

% Картинки
\newcommand{\imgh}[3] {
\begin{figure}[t!]
    \center{\includegraphics[height=#1]{inc/img/#2}}
    \caption{#3}
    \label{img:#2}
\end{figure}
}
\newcommand{\imgw}[3] {
\begin{figure}[h!]
    \center{\includegraphics[width=#1]{inc/img/#2}}
    \caption{#3}
    \label{img:#2}
\end{figure}
}
\newcommand{\imgs}[4]
{
	\begin{figure}[#2]
		\center{\includegraphics[scale=#3]{inc/img/#1}}
		\caption{#4}
		\label{img:#1}
	\end{figure}
}
\newcommand{\boximg}[3] {
\begin{figure}[h]
    \center{\fbox{\includegraphics[height=#1]{inc/img/#2}}}
    \caption{#3}
    \label{img:#2}
\end{figure}
}

\newcommand{\aasection}[2] {
    \vspace{20mm}
    {\let\clearpage\relax \chapter{#1}\label{#2}}
}

\newcommand{\aaunnumberedsection}[2] {
    \vspace{20mm}
    \addcontentsline{toc}{chapter}{#1}
    {\let\clearpage\relax \chapter*{#1}\label{#2}}
}

\usepackage{csvsimple}
\usepackage{dirtree}

\usepackage{listings}
\usepackage{listingsutf8}
\lstset{
	basicstyle=\small\ttfamily,			% размер и начертание шрифта для подсветки кода
	frame=single,							% рисовать рамку вокруг кода
	numbers=none,						% где поставить нумерацию строк (слева\справа)
	tabsize=4,							% размер табуляции по умолчанию равен 4 пробелам
	breaklines=true,
	inputencoding=utf8/koi8-r,
}

\lstset{
	keepspaces=true,
	extendedchars=\true,
	inputencoding=utf8x,
%	escapechar={|}
	alsoletter={"},
	mathescape=false,
	deletestring=[b]",
}

\newcommand{\listingfilec}[5]{
	\lstinputlisting[
	float=h!,
	frame=single,						% рисовать рамку вокруг кода
	numbers=none,						% где поставить нумерацию строк (слева\справа)
	abovecaptionskip=-5pt,
	caption={#4},
	label={lst:#2},
	language={#3},
	#5,
	]{inc/lst/#1}
}